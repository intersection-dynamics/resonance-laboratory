\documentclass[11pt,twocolumn]{article}

% Packages
\usepackage[utf8]{inputenc}
\usepackage[T1]{fontenc}
\usepackage{amsmath,amssymb,amsfonts}
\usepackage{graphicx}
\usepackage{xcolor}
\usepackage{hyperref}
\usepackage{cleveref}
\usepackage{booktabs}
\usepackage{pgfplots}
\pgfplotsset{compat=1.18}
\usepackage{physics}
\usepackage[margin=1in]{geometry}

% Custom commands
\newcommand{\hilbert}{\mathcal{H}}
\newcommand{\ham}{\hat{H}}
\newcommand{\bra}[1]{\langle #1 |}
\newcommand{\ket}[1]{| #1 \rangle}
\newcommand{\braket}[2]{\langle #1 | #2 \rangle}
\newcommand{\expect}[1]{\langle #1 \rangle}

\title{Hilbert Space Structure and the Emergence of Exchange Statistics:\\
A Computational Investigation}

\author{Ben\\
\textit{Independent Researcher, Florida}}

\date{November 2025}

\begin{document}

\maketitle

\begin{abstract}
We present computational investigations into correspondences between Hilbert space structure and features of the Standard Model. We examine the well-known result that $n$-dimensional complex vector spaces carry $\text{SU}(n)$ symmetry and explore whether this mathematical fact, combined with local unitary dynamics, can provide pedagogical insight into gauge structure. We report the explicit \textbf{falsification} of an information-theoretic hypothesis: numerical experiments demonstrate that symmetric and antisymmetric two-particle states exhibit identical copyability under partial-SWAP and CNOT protocols, ruling out ``resistance to copying'' as the origin of fermionic statistics. We verify computationally that exchange phase stability under local Hamiltonians is consistent with the known $\mathbb{Z}_2$ topology of three-dimensional configuration space. The work illustrates the scientific method applied to foundational questions, including hypothesis generation, quantitative testing, and falsification. We discuss limitations, propose experimental tests, and outline future directions.
\end{abstract}

%==============================================================================
\section{Introduction}
%==============================================================================

The Standard Model of particle physics describes electromagnetic, weak, and strong interactions through the gauge group $\text{U}(1) \times \text{SU}(2) \times \text{SU}(3)$. The spin-statistics theorem, proved within relativistic quantum field theory, establishes that integer-spin particles obey Bose-Einstein statistics while half-integer-spin particles obey Fermi-Dirac statistics~\cite{Streater1964,Weinberg1995}. These results are well-established; the present work does not claim to derive them anew.

This paper reports on a computational research program with two goals. First, we investigate whether a minimal ``substrate'' model---a tensor-product Hilbert space with local unitary dynamics---can serve as a pedagogical framework for understanding correspondences between Hilbert space geometry and physical structures. Second, we subject specific hypotheses to quantitative tests, reporting both confirmations and \textbf{falsifications}.

The falsification results are central to this work. We initially hypothesized that particle statistics might originate from information-theoretic properties---specifically, that fermionic states might ``resist copying'' in ways that bosonic states do not. Our numerical experiments definitively rule out this hypothesis: symmetric and antisymmetric states exhibit identical copyability metrics across multiple protocols.

%==============================================================================
\section{Substrate Model and Methods}
%==============================================================================

\subsection{Model Definition}

We consider a lattice of $N$ sites with local Hilbert space $\hilbert_{\text{site}} \cong \mathbb{C}^d$ at each site, giving total Hilbert space
\begin{equation}
\hilbert = \bigotimes_{i=1}^{N} \hilbert_{\text{site}}.
\end{equation}
Dynamics are generated by a local Hamiltonian
\begin{equation}
\ham = \sum_i h_i + \sum_{\langle i,j \rangle} h_{ij},
\end{equation}
where $\langle i,j \rangle$ denotes nearest neighbors. For most simulations, we use a hopping Hamiltonian
\begin{equation}
\ham = -t \sum_{\langle i,j \rangle} \left( \ket{i}\bra{j} + \ket{j}\bra{i} \right)
\end{equation}
in the single-excitation subspace.

\subsection{Exchange States}

For two-particle sectors, we define exchange states
\begin{equation}
\ket{\psi_\phi} = \frac{1}{\sqrt{2}} \left( \ket{10} + e^{i\phi} \ket{01} \right),
\end{equation}
where $\phi$ is the exchange phase. The symmetric state ($\phi = 0$) and antisymmetric state ($\phi = \pi$) are eigenstates of the exchange operator $P_{12}$ with eigenvalues $+1$ and $-1$ respectively.

\subsection{Observables}

We measure exchange fidelity
\begin{equation}
F_\phi(t) = \left| \braket{\psi_\phi(0)}{\psi_\phi(t)} \right|
\end{equation}
and long-time stability
\begin{equation}
S(\phi; L) = \frac{1}{T} \int_0^T F_\phi(t; L) \, dt.
\end{equation}

%==============================================================================
\section{Spin-Statistics via Dynamics}
%==============================================================================

\subsection{Topological Background}

The configuration space of two indistinguishable particles in $d$ dimensions has fundamental group $\mathbb{Z}$ for $d = 2$ and $\mathbb{Z}_2$ for $d \geq 3$~\cite{Laidlaw1971,Leinaas1977}. This implies that in three dimensions, the exchange phase $\phi$ must satisfy $e^{2i\phi} = 1$, giving $\phi \in \{0, \pi\}$ as the only consistent values.

\subsection{Numerical Results}

\begin{figure}[t]
\centering
\begin{tikzpicture}
\begin{axis}[
    width=\columnwidth,
    height=5cm,
    xlabel={Time $t$},
    ylabel={Fidelity $F_\phi(t)$},
    legend pos=south east,
    grid=major,
    ymin=0, ymax=1.1
]
% Data from fidelity_L32_*.dat files
\addplot[blue, thick] table[x index=0, y index=1] {data/fidelity_L32_phi0.dat};
\addplot[red, thick, dashed] table[x index=0, y index=1] {data/fidelity_L32_phipi.dat};
\addplot[green!60!black, thick, dotted] table[x index=0, y index=1] {data/fidelity_L32_phihalf.dat};
\legend{$\phi = 0$, $\phi = \pi$, $\phi = \pi/2$}
\end{axis}
\end{tikzpicture}
\caption{Exchange fidelity $F_\phi(t)$ for $L = 32$ sites. The $\phi = 0$ (bosonic) and $\phi = \pi$ (fermionic) sectors remain stable, while $\phi = \pi/2$ decays.}
\label{fig:fidelity}
\end{figure}

\Cref{fig:fidelity} shows the exchange fidelity versus time for different phases. The $\phi = 0$ and $\phi = \pi$ sectors maintain high fidelity, while intermediate phases decay.

\begin{figure}[t]
\centering
\begin{tikzpicture}
\begin{axis}[
    width=\columnwidth,
    height=5cm,
    xlabel={Exchange phase $\phi / \pi$},
    ylabel={Stability $S(\phi)$},
    legend pos=south east,
    grid=major,
    ymin=0, ymax=1.1,
    xtick={0, 0.25, 0.5, 0.75, 1},
    xticklabels={$0$, $\pi/4$, $\pi/2$, $3\pi/4$, $\pi$}
]
\addplot[blue, thick, mark=*] table[x expr=\thisrowno{0}/3.14159, y index=1] {data/stability_vs_phi_L8.dat};
\addplot[green!60!black, thick, mark=square] table[x expr=\thisrowno{0}/3.14159, y index=1] {data/stability_vs_phi_L16.dat};
\addplot[red, thick, mark=triangle] table[x expr=\thisrowno{0}/3.14159, y index=1] {data/stability_vs_phi_L32.dat};
\legend{$L=8$, $L=16$, $L=32$}
\end{axis}
\end{tikzpicture}
\caption{Long-time stability $S(\phi)$ versus exchange phase. Sharp peaks at $\phi = 0$ and $\phi = \pi$, with suppression of intermediate phases increasing with system size.}
\label{fig:stability}
\end{figure}

\Cref{fig:stability} shows the stability metric versus exchange phase. The peaks at $\phi = 0$ and $\phi = \pi$ sharpen with increasing system size, consistent with the $\mathbb{Z}_2$ topological constraint becoming exact in the thermodynamic limit.

\begin{figure}[t]
\centering
\begin{tikzpicture}
\begin{semilogyaxis}[
    width=\columnwidth,
    height=5cm,
    xlabel={System size $L$},
    ylabel={Stability $S(\pi/2)$},
    grid=major
]
\addplot[black, thick, mark=*] table[x index=0, y index=1] {data/finite_size_scaling.dat};
\end{semilogyaxis}
\end{tikzpicture}
\caption{Finite-size scaling of $S(\pi/2)$. The intermediate phase is exponentially suppressed with system size.}
\label{fig:scaling}
\end{figure}

\Cref{fig:scaling} demonstrates finite-size scaling: the stability of the intermediate phase $\phi = \pi/2$ decreases approximately exponentially with system size, from $S \approx 0.49$ at $L = 4$ to $S \approx 0.09$ at $L = 64$.

%==============================================================================
\section{Falsification of the Copyability Hypothesis}
%==============================================================================

\subsection{The Hypothesis}

We hypothesized that fermionic statistics might have an information-theoretic origin: if fermionic states are harder to copy than bosonic states, this could connect the Pauli exclusion principle to the no-cloning theorem.

\subsection{Protocol}

We tested copyability using a partial-SWAP protocol: starting with a state $\ket{\psi}$ in the symmetric or antisymmetric sector, we applied progressive $\sqrt{\text{SWAP}}$ operations and measured:
\begin{itemize}
\item Purity of the reduced state
\item Correlation with the original
\item Copyability metric: $\sqrt{\text{purity} \times \text{correlation}}$
\end{itemize}

\subsection{Results}

\begin{figure}[t]
\centering
\begin{tikzpicture}
\begin{axis}[
    width=\columnwidth,
    height=5cm,
    xlabel={Protocol step},
    ylabel={Copyability metric},
    legend pos=north east,
    grid=major
]
\addplot[blue, thick] table[x index=0, y index=1] {data/copyability_comparison.dat};
\addplot[red, thick, dashed] table[x index=0, y index=2] {data/copyability_comparison.dat};
\legend{Symmetric, Antisymmetric}
\end{axis}
\end{tikzpicture}
\caption{Copyability comparison between symmetric (bosonic) and antisymmetric (fermionic) states. \textbf{The curves are identical to six decimal places.} The hypothesis is falsified.}
\label{fig:copyability}
\end{figure}

The results (\Cref{fig:copyability}) are unambiguous:
\begin{itemize}
\item Symmetric state: copyability = 0.7071
\item Antisymmetric state: copyability = 0.7071
\item Difference: $< 10^{-6}$
\end{itemize}

\textbf{The copyability hypothesis is falsified.} Symmetric and antisymmetric states are informationally indistinguishable. The only property that differs is the geometric phase under exchange.

%==============================================================================
\section{Confinement Toy Model}
%==============================================================================

We constructed a three-quark toy model with SU(3) color and Hamiltonian $\ham = \kappa \cdot C_2$, where $C_2$ is the quadratic Casimir operator for total color.

\begin{figure}[t]
\centering
\begin{tikzpicture}
\begin{axis}[
    width=\columnwidth,
    height=5cm,
    xlabel={Confinement parameter $\kappa$},
    ylabel={Energy},
    legend pos=north west,
    grid=major
]
\addplot[blue, thick, mark=*] table[x index=0, y index=1] {data/confinement_spectrum.dat};
\addplot[red, thick, mark=square] table[x index=0, y index=2] {data/confinement_spectrum.dat};
\legend{Singlet, Non-singlet}
\end{axis}
\end{tikzpicture}
\caption{Energy of color-singlet versus non-singlet states as a function of confinement parameter $\kappa$. The gap grows linearly: $\Delta E = 3\kappa$.}
\label{fig:confinement}
\end{figure}

\Cref{fig:confinement} shows that color-singlet states have $E = 0$ for all $\kappa$, while non-singlet states have $E = 3\kappa$. This demonstrates the energetic preference for color-neutral bound states, reminiscent of QCD confinement.

%==============================================================================
\section{Gauge Group Correspondence}
%==============================================================================

The symmetry group of an $n$-dimensional complex vector space with preserved inner product is $\text{U}(n) = \text{U}(1) \times \text{SU}(n)$. This gives:

\begin{center}
\begin{tabular}{ccc}
\toprule
Dimension & Group & Generators \\
\midrule
1 & U(1) & 0 \\
2 & SU(2) & 3 \\
3 & SU(3) & 8 \\
\bottomrule
\end{tabular}
\end{center}

The product $\text{U}(1) \times \text{SU}(2) \times \text{SU}(3)$ matches the Standard Model gauge group. We verified the Lie algebra structure computationally by constructing the generalized Gell-Mann matrices and confirming commutation relations.

This correspondence is \textit{suggestive} but requires caution: the mathematical coincidence does not explain why nature employs these specific dimensions.

%==============================================================================
\section{Proposed Experimental Tests}
%==============================================================================

\subsection{Exchange Phase Stability}

\textbf{Platform:} Rydberg atom arrays, superconducting qubits, or trapped ions.

\textbf{Protocol:} Prepare two-excitation states with controlled exchange phase $\phi$. Allow free evolution under native hopping Hamiltonian. Measure fidelity versus time.

\textbf{Prediction:} $\phi = 0$ and $\phi = \pi$ sectors maintain high fidelity; intermediate phases show decay increasing with system size.

\subsection{Copyability Verification}

\textbf{Platform:} Photonic circuits or superconducting qubits with high-fidelity two-qubit gates.

\textbf{Protocol:} Prepare symmetric and antisymmetric Bell-like states. Implement partial-SWAP copying protocol. Perform state tomography.

\textbf{Prediction:} Copyability metrics identical for both state types within experimental error.

\subsection{Dimensional Dependence}

\textbf{Platform:} 2D systems (FQHE, topological superconductors).

\textbf{Prediction:} Intermediate phases ($\phi \neq 0, \pi$) should be stable in 2D, unstable in 3D. This tests the dimensional dependence of the topological constraint.

%==============================================================================
\section{Limitations}
%==============================================================================

We explicitly state what this work does \textbf{not} accomplish:

\begin{enumerate}
\item \textbf{Mathematical identity vs.\ physical derivation:} The correspondence between $n$-dimensional spaces and SU$(n)$ is mathematics, not physics. We have not explained why nature uses dimensions 1, 2, 3.

\item \textbf{No dynamics:} We provide no mechanism for symmetry breaking, coupling constants, or mass spectrum.

\item \textbf{Non-relativistic:} We do not derive Lorentz invariance.

\item \textbf{Known results:} The $\mathbb{Z}_2$ topology and SU(2) double-cover are textbook material. Our contribution is computational verification, not discovery.

\item \textbf{Finite-size effects:} Simulations used lattices up to 256 sites. Larger-scale studies would strengthen conclusions.
\end{enumerate}

%==============================================================================
\section{Future Directions}
%==============================================================================

\begin{enumerate}
\item \textbf{Relativistic extension:} Incorporate Lorentz symmetry through discrete models or causal set formulations.

\item \textbf{Origin of dimensions 1, 2, 3:} Investigate stability, information-theoretic optimality, or connections to spatial dimensionality.

\item \textbf{Emergent gauge dynamics:} Derive gauge bosons as propagating degrees of freedom from entanglement structure.

\item \textbf{Gravity:} Explore whether spacetime geometry can emerge from entanglement within this framework.
\end{enumerate}

%==============================================================================
\section{Conclusion}
%==============================================================================

This work makes three contributions:

\begin{enumerate}
\item \textbf{Infrastructure:} Computational tools for exploring Hilbert space / Standard Model correspondences.

\item \textbf{Falsification:} The copyability hypothesis is ruled out---symmetric and antisymmetric states are informationally identical.

\item \textbf{Verification:} Exchange phase stability is consistent with $\mathbb{Z}_2$ topology in 3D.
\end{enumerate}

The falsification result exemplifies the scientific method: hypothesis, test, rejection. The distinction between bosons and fermions is geometric, not information-theoretic.

We do not claim to have derived the Standard Model. What remains unexplained---why Hilbert space, why dimensions 1-2-3, why 3D space---may be fundamental or may yield to future investigation.

\section*{Acknowledgments}

This work was conducted independently over approximately twenty years. The author thanks Claude (Anthropic) for computational collaboration including code development, hypothesis testing, and critical discussion. The author also thanks Dr.\ Xiao-Gang Wen (MIT) for encouragement.

\begin{thebibliography}{10}

\bibitem{Streater1964}
R.~F. Streater and A.~S. Wightman, \textit{PCT, Spin and Statistics, and All That} (Princeton University Press, 1964).

\bibitem{Weinberg1995}
S.~Weinberg, \textit{The Quantum Theory of Fields}, Vol.~1 (Cambridge University Press, 1995).

\bibitem{Laidlaw1971}
M.~G.~G. Laidlaw and C.~M. DeWitt, Feynman functional integrals for systems of indistinguishable particles, Phys.\ Rev.\ D \textbf{3}, 1375 (1971).

\bibitem{Leinaas1977}
J.~M. Leinaas and J.~Myrheim, On the theory of identical particles, Nuovo Cimento B \textbf{37}, 1 (1977).

\bibitem{Sakurai2017}
J.~J. Sakurai and J.~Napolitano, \textit{Modern Quantum Mechanics}, 2nd ed.\ (Cambridge University Press, 2017).

\bibitem{Wen2004}
X.-G. Wen, \textit{Quantum Field Theory of Many-Body Systems} (Oxford University Press, 2004).

\bibitem{Wilczek1982}
F.~Wilczek, Quantum mechanics of fractional-spin particles, Phys.\ Rev.\ Lett.\ \textbf{49}, 957 (1982).

\bibitem{Nielsen2000}
M.~A. Nielsen and I.~L. Chuang, \textit{Quantum Computation and Quantum Information} (Cambridge University Press, 2000).

\end{thebibliography}

\end{document}
